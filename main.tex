\documentclass[12pt]{article}
\usepackage{ArtigoIFPE}

\title{COMO PREPARAR UM ARTIGO CIENTÍFICO (EM CAIXA ALTA): subtítulo do trabalho se houver (em caixa baixa)}
\titleEng{TÍTULO EM INGLÊS (OPCIONAL, CAIXA ALTA): subtítulo (caixa baixa)}

\autora{Nome do autor 1}
\emaila{Email do autor 1}
%\autorb{Nome do autor 2}
%\emailb{Email do autor 2}
% Aceita até 6 autores (autora, autorb, autorc, ... , autorf) e seus emails
\orientador{Orientador}
\emailOrientador{Email do orientador}

\campus{Pesqueira}
\curso{de Bacharelado em Engenharia Elétrica}
\data{07 de março de 2024}

\begin{document}

\maketitle

\thispagestyle{plain}

\section*{Resumo}

\noindent O resumo é um elemento obrigatório apresentado em um único parágrafo. Indica os principais assuntos abordados, apresentando o(s) objetivo(s), o método, os resultados e as considerações finais sem citação bibliográfica ou abreviação não definida. Na sua extensão deve conter no mínimo 100 e no máximo 250 palavras. Segundo a norma da Associação Brasileira de Normas Técnicas (ABNT) NBR 6028 (2003, seção 3.3.4) ``devem-se evitar símbolos e contrações que não sejam de uso corrente'' como também ``fórmulas, equações, diagramas etc., que não sejam absolutamente necessários; quando seu emprego for imprescindível, defini-los na primeira vez que aparecerem''. 

\palavraschave{Palavra 1. Palavra 2. Palavra 3.}

\section*{Abstract}

\noindent Modelo modelo modelo modelo modelo modelo modelo modelo modelo modelo modelo modelo modelo modelo modelo modelo modelo modelo modelo modelo modelo modelo modelo modelo modelo modelo modelo modelo modelo modelo modelo modelo modelo modelo modelo modelo modelo modelo.

\keywords{Palavra 1. Palavra 2. Palavra 3.}

\vspace*{20pt} \hrule height 1.5pt

\section{Introdução}

Teste teste teste teste teste teste teste teste teste teste teste teste teste teste teste teste teste teste teste teste teste teste teste teste teste teste teste teste teste teste teste teste teste teste teste teste teste teste teste teste teste teste teste teste teste teste teste teste teste teste teste teste teste teste teste teste teste teste teste teste. Exemplo de citação textual, conforme \textcite{world2023}, e exemplo de citação direta curta encerrando parágrafo \parencite{world2023}.
% No arquivo compilado, fica uma marcação em verde indicando que é um link. Mas não se preocupe, pois, quando baixa o PDF, as marcações somem.

Teste teste teste teste teste teste teste teste teste teste teste teste teste teste teste teste teste teste teste teste teste teste teste teste teste teste teste teste teste teste teste teste teste teste teste teste teste teste teste teste teste teste teste teste teste teste teste teste teste teste teste teste teste teste teste teste teste teste teste teste teste teste teste.

Utilize o ambiente \textit{citacao} para incluir citações com mais de três linhas.

\begin{citacao}
Exemplo de citação com mais de três linhas. Exemplo de citação com mais de três linhas. Exemplo de citação com mais de três linhas. Exemplo de citação com mais de três linhas. Exemplo de citação com mais de três linhas. Exemplo de citação com mais de três linhas. Exemplo de citação com mais de três linhas. 
\end{citacao}

Um exemplo para equação ou fórmula pode ser visto a seguir.
\begin{equation} \label{eu_eqn}
e^{\pi i} + 1 = 0
\end{equation}

Se forem várias linhas \footnote{Mais informações sobre alinhamento de equações: \url{https://www.overleaf.com/learn/latex/Aligning_equations_with_amsmath}}:
\begin{equation}
\label{eq1}
\begin{split}
A & = \frac{\pi r^2}{2} \\
 & = \frac{1}{2} \pi r^2
\end{split}
\end{equation}



Ilustrações: Qualquer que seja o tipo de ilustração, esta deve ser precedida de sua palavra designativa (neste arquivo estão definidos os ambientes: \textit{figura}, \textit{diagrama}, \textit{fluxograma}, \textit{quadro}, \textit{tabela}, \textit{grafico} e \textit{mapa}). Imediatamente após a ilustração, deve-se indicar a fonte consultada (elemento obrigatório, mesmo que seja produção do próprio autor) conforme a ABNT NBR 10520, legenda, notas e outras informações necessárias à sua compreensão (se houver). A ilustração deve ser citada no texto e inserida o mais próximo possível do trecho a que se refere. A citação pode ser utilizando o comando \textit{autoref}, ex.: conforme pode ser visto no \autoref{dia:VSGDSC}.

\begin{diagrama}
    \caption{Blocos do VS-GDSC.}
    \centering
    \includegraphics[width=0.8\linewidth]{Imagens/diagramaBlocos.eps}\\
    \XPT Fonte: \textcite{batista2015}
    \label{dia:VSGDSC}
\end{diagrama}

Tabelas: Devem ser citadas no texto, inseridas o mais próximo possível do trecho a que se referem, e padronizadas conforme as Normas de apresentação tabular do IBGE. Um exemplo pode ser visto na \autoref{tab:Residentes}. Deve-se indicar a fonte consultada (elemento obrigatório, mesmo que seja produção do próprio autor), de acordo com a ABNT NBR 10520.

\begin{tabela}[hbt]
    \caption{Pessoas residentes em domicílios particulares, por sexo e situação do domicílio - Brasil - 1980}
    \centering
    \vspace{-6pt} % Este comando aproxima ou afasta os "floats" dos textos. Neste caso, estamos aproximando a Legenda da Tabela
    \begin{tabular}{lccc}
    \specialrule{2pt}{0pt}{1pt}
    Situação do Total & Total & Mulheres & Homens \\
    \specialrule{1pt}{1pt}{1pt}
    Total & 117.960.301 & 59.595.332 & 58.364.969\\
    Urbana & 79.972.931 & 41.115.439 & 38.857.492 \\
    Rural & 37.987.370 & 18.479.893 & 19.507.477 \\
    \specialrule{2pt}{1pt}{0pt}
    \end{tabular}\\
    \vspace{3pt}
    \XPT Fonte: IBGE (2013)
    \label{tab:Residentes}
\end{tabela}

 Teste teste teste teste teste teste teste teste teste teste teste teste teste teste teste teste teste teste teste teste teste teste teste teste teste teste teste teste teste teste teste teste teste teste teste teste teste teste teste teste teste teste teste teste teste teste teste teste teste teste teste teste teste.

 \begin{comment}
Comentários longos podem ser inseridos no ambiente comment. Este texto não será processado pelo compilador e não irá aparecer no PDF final.
 \end{comment}

\section{Desenvolvimento}

  Durante a revisão é comum marcar partes do texto. \hl{Usar o comando \textit{hl} para marcar o texto}. Teste teste teste teste teste teste teste teste teste teste teste teste teste teste teste teste teste teste teste teste teste teste teste teste teste teste teste teste teste teste teste teste teste teste teste teste teste teste teste teste teste teste teste teste teste teste teste teste teste teste teste teste teste teste teste teste teste teste teste

  \subsection{Tópico 1}

  Teste teste teste teste teste teste teste teste teste teste teste teste teste teste teste teste teste teste teste teste teste teste teste teste teste teste teste teste teste teste teste teste teste teste teste teste teste teste teste teste teste teste teste teste teste teste teste teste teste. Exemplo de listagem. Este trabalho de conclusão de curso foi desenvolvido em quatro etapas:
\begin{itemize}
\item Escolha do microcontrolador;
\item Desenvolvimento de algoritmo para receber e segmentar o \textit{status} do módulo SIM900;
\item Projeto do circuito eletrônico para \textit{reset} do módulo SIM900;
\item Design e confecção da PCI na CNC \textit{Router}.
\end{itemize}

  \subsection{Tópico 2}

 Teste teste teste teste teste teste teste teste teste teste teste teste teste teste teste teste teste teste teste teste teste teste teste teste teste teste teste teste teste teste teste teste teste teste teste teste teste teste teste teste teste teste teste teste teste teste teste teste teste teste teste teste teste.
 
 Para referenciar um apêndice ou anexo, podem ser utilizados os comandos \textit{ref}, conforme exemplo do \ref{ape:ProgramaPython} ou \ref{anx:ListaAT}.

 \subsubsection{Subtópico}

  Teste teste teste teste teste teste teste teste teste teste teste teste teste teste teste teste teste teste teste teste teste teste teste teste teste teste teste teste teste teste teste teste teste teste teste teste teste teste teste teste teste teste teste teste teste teste teste teste teste teste teste teste teste teste teste teste teste teste teste

\section{Metodologia}

  Teste teste teste teste teste teste teste teste teste teste teste teste teste teste teste teste teste teste teste teste teste teste teste teste teste teste teste teste teste teste teste teste teste teste teste teste teste teste teste teste teste teste teste teste teste teste teste teste teste teste teste teste teste teste teste teste teste teste teste teste teste teste teste teste teste teste teste teste teste teste teste teste teste teste teste teste teste teste teste teste teste teste teste teste teste teste.

  \section{Resultados e análise}

 Teste teste teste teste teste teste teste teste teste teste teste teste teste teste teste teste teste teste teste teste teste teste teste teste teste teste teste teste teste teste teste teste teste teste teste teste teste teste teste teste teste teste teste teste teste teste teste teste teste teste teste teste teste.

\section{Considerações finais / Conclusões}
 
  Teste teste teste teste teste teste teste teste teste teste teste teste teste teste teste teste teste teste teste teste teste teste teste teste teste teste teste teste teste teste teste teste teste teste teste teste teste teste teste teste teste teste teste teste teste teste teste teste teste teste teste teste teste teste teste teste teste teste teste

  \printbibliography[title={REFERÊNCIAS}]

\reiniciaAlfa % Reinicia numeração para os Apêndices

\apendice{Exemplo de programa em Python}
\label{ape:ProgramaPython}

% Mais informações sobre como inserir códigos fonte: https://pt.overleaf.com/learn/latex/Code_listing
\begin{lstlisting}[language=Python,caption=Exemplo.]
import numpy as np
    
def incmatrix(genl1,genl2):
    m = len(genl1)
    n = len(genl2)
    M = None #to become the incidence matrix
    VT = np.zeros((n*m,1), int)  #dummy variable
    
    #compute the bitwise xor matrix
    M1 = bitxormatrix(genl1)
    M2 = np.triu(bitxormatrix(genl2),1) 

    for i in range(m-1):
        for j in range(i+1, m):
            [r,c] = np.where(M2 == M1[i,j])
            for k in range(len(r)):
                VT[(i)*n + r[k]] = 1;
                VT[(i)*n + c[k]] = 1;
                VT[(j)*n + r[k]] = 1;
                VT[(j)*n + c[k]] = 1;
                
                if M is None:
                    M = np.copy(VT)
                else:
                    M = np.concatenate((M, VT), 1)
                
                VT = np.zeros((n*m,1), int)
    
    return M
\end{lstlisting}

\reiniciaAlfa % Reinicia numeração para os Anexos

\anexo{Lista de comandos AT utilizados}
\label{anx:ListaAT}

\begin{itemize}
    \item AT: Comando de verificação básica de comunicação, onde o módulo geralmente responde com "OK";
    \item AT+CCID: Verifica o número do cartão SIM inserido no módulo. O módulo responderá com o número ICCID (\textit{Integrated Circuit Card Identifier}) do cartão SIM, que é um identificador único do cartão;
    \item AT+CPIN: Verifica o status do PIN (\textit{Personal Identification Number}) do cartão SIM;
    \item AT+CREG: Consulta o status de registro na rede;
    \item AT+CSQ: Consulta a qualidade do sinal da rede;
    \item AT+CMGF: Define o modo de mensagem (texto ou PDU) para envio e recebimento de SMS;
    \item AT+CMGS: Envia uma mensagem de texto;
    \item AT+CMGL: Lista mensagens SMS armazenadas na memória do módulo;
    \item AT+CMGR: Lê uma mensagem SMS específica;
    \item AT+CMGD: Exclui uma mensagem SMS;
    \item ATD: Inicia uma chamada telefônica;
    \item ATH: Encerra uma chamada telefônica;
    \item AT+CGATT: Ativa ou desativa o serviço GPRS;
    \item AT+CGDCONT: Define os parâmetros da conexão GPRS;
    \item AT+CIICR: Inicia a conexão GPRS;
    \item AT+CIPSTART: Inicia uma conexão TCP ou UDP;
    \item AT+CIPSEND: Envia dados por meio da conexão TCP ou UDP;
    \item AT+CIPCLOSE: Fecha uma conexão TCP ou UDP;
    \item AT+SAPBR: Configura e gerencia a conexão GPRS.
\end{itemize}

\end{document}
